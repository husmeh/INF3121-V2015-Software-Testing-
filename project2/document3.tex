\documentclass[a4paper,10pt]{article}
\usepackage[utf8]{inputenc}

\usepackage[T1]{fontenc}
\usepackage{color}

\usepackage[english]{babel}
\usepackage[english]{isodate}

% Hyperlinks in PDF: \href{url}{linktext}
\definecolor{linkcolor}{rgb}{0,0,0.4}
\usepackage[%
    colorlinks=true,
    linkcolor=linkcolor,
    urlcolor=linkcolor,
    citecolor=black,
    filecolor=black,
]{hyperref}

%opening
\title{INF3121 Assignment 2, Document 3 \\ Explanation of Manual Tests That Could Not Be Automated}

\author{
Even Langfeldt Friberg\footnote{\href{mailto:evenlf@student.matnat.uio.no}{\nolinkurl{evenlf@student.matnat.uio.no}}.}
\and Husein Mehmedagic\footnote{\href{mailto:huseinm@student.matnat.uio.no}{\nolinkurl{huseinm@student.matnat.uio.no}}.}
}

\begin{document}

\maketitle

\section{Test Case: CreateUserAccount.file}
The value of \texttt{id=wpName2} is \textit{husven\_test3}, but since we already created this user when we performed the manual test, this username is 
no longer unique. One could of course 
change this value to something unique, but the automated test would still fail because the value of \texttt{id=wpCatchaWord} was \textit{aryanbusy} when we first 
interpreted the captcha, and this is completely unlikely to be the answer to a new captcha given.

One could probably make an automated test that copied the source of the captcha image, made use of an online decaptcha service, copied the computed captcha word 
and pasted it into \texttt{id=wpCaptchaWord}, but we did not find such a service available freely.

\end{document}
