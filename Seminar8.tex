\documentclass[12pt,a4paper,norsk]{article}

\usepackage{fancyvrb}
\usepackage[T1]{fontenc}
\usepackage[utf8]{inputenc}
\usepackage{fancyvrb}
\usepackage{framed}
\usepackage{color}
\usepackage[colorlinks]{hyperref}
\hypersetup{linkcolor=DarkRed}
\providecommand{\shadedwbar}{}
\definecolor{shadecolor}{rgb}{0.87843, 0.95686, 1.0}
\renewenvironment{shadedwbar}{
\def\FrameCommand{\color[rgb]{0.7,     0.95686, 1}\vrule width 1mm\normalcolor\colorbox{shadecolor}}\FrameRule0.6pt
\MakeFramed {\advance\hsize-2mm\FrameRestore}\vskip3mm}{\vskip0mm\endMakeFramed}
\providecommand{\shadedquoteBlueBar}{}
\renewenvironment{shadedquoteBlueBar}[1][]{
\bgroup\rmfamily
\fboxsep=0mm\relax
\begin{shadedwbar}
\list{}{\parsep=-2mm\parskip=0mm\topsep=0pt\leftmargin=2mm
\rightmargin=2\leftmargin\leftmargin=4pt\relax}
\item\relax}
{\endlist\end{shadedwbar}\egroup}


\title{Seminar 8 for Lecture 7}
\author{Husein Mehmedagic <huseinm@student.matnat.uio.no>}
\date{\today}


\begin{document}
\maketitle
\noindent \section {Question 1:} Wich of the following is a a purpose of HCI testing?\\
\underline {\textbf{Answer: c}}. It tests that the software  is understandable:\\
\indent Purpose is to be easy to learn, use and remember. That it is understandable and satisfactory to use.\\
\noindent Coments:\\
\framebox(450,100){}\\

\noindent \section{ Question 2:} Which components constitute the HCI framework?\\
\underline{\textbf{Answer: d}}. Interface standards, usability, interface dynamics, aesthetics.
\subsection{Interface standards :} \\
\indent \textbf{Interface standards are constituted by:}\\
\indent - Best practices\\
\indent - Consistent behavior and design\\
\indent - Decrease work load\\
\indent - Faster development\\
\indent - Be consistent

\subsection{Usability}\\
\textbf{Usability means:}\\
\indent - Effectiveness\\
\indent - Efficiency\\
\indent - Satisfaction\\

\noindent Note: The key is to understand the target and users and their needs and create a usercentric design.\\

\subsection{Interface dynamics}\\
\textbf{An interface(whether visual or API) has to be designed in such way that it is:}\\
\indent - Responsive and fast \\
\indent - Adaptable to the users needs and context\\
\indent - Empowers the user\\
\indent - Captivating\\
\indent - Dynamic\\

\subsection{Aesthetics}\\
\indent - Responsible for the first impression\\
\indent - Modern, fresh, appealing design\\
\indent - Recognition of a company's applications\\
\indent - A company's graphical profile\\
\noindent Coments:\\
\framebox(450,100){}\\

\section {Question 3:} Which of the following represents interface dynamics principles?\\
\underline{\textbf{Answer: a}} Software has to be responsive, fast and adaptable to the users needs & user context\\
\noindent Se subsection 2.3 Interface Dynamics.\\

\noindent Coments:\\
\framebox(450,100){}\\

\section {Question 4:} Which of the following is a good practice when using system alerts?\\
\underline{\textbf{Answer: c}} Never use error codes, jargon or technical terms - speak the users language.\\
\indent - Never use capital letters or exclamation mark - you scream at the user.\\
\noindent Note:\\
\indent - Keep the message short, people do not read.\\
\indent - Use the right action buttons.\\
\indent - Errors and warnings are never OK, use Close\\
\indent - To have Yes, No and Cancle for a question is confusing\\

\noindent Coments:\\
\framebox(450,100){}\\

\section {Question 5: Open-end question(Desktop application)} Can you tell me a few usability issues with the following Application?\\

\noindent Coments:\\
\framebox(450,100){}\\

\section {Question 5: Open-end question(Webpage)} Can you tell me a few usability issues with the following webpage?The Yale University School of Art: \url{http//art.yale.edu/} \\

\noindent Coments:\\
\framebox(450,100){}\\

\end{document}