\documentclass[12pt,a4paper,norsk]{article}

\usepackage{fancyvrb}
\usepackage[T1]{fontenc}
\usepackage[utf8]{inputenc}
\usepackage{fancyvrb}
\usepackage{framed}
\usepackage{color}
\usepackage[colorlinks]{hyperref}
\hypersetup{linkcolor=DarkRed}
\providecommand{\shadedwbar}{}
\definecolor{shadecolor}{rgb}{0.87843, 0.95686, 1.0}
\renewenvironment{shadedwbar}{
\def\FrameCommand{\color[rgb]{0.7,     0.95686, 1}\vrule width 1mm\normalcolor\colorbox{shadecolor}}\FrameRule0.6pt
\MakeFramed {\advance\hsize-2mm\FrameRestore}\vskip3mm}{\vskip0mm\endMakeFramed}
\providecommand{\shadedquoteBlueBar}{}
\renewenvironment{shadedquoteBlueBar}[1][]{
\bgroup\rmfamily
\fboxsep=0mm\relax
\begin{shadedwbar}
\list{}{\parsep=-2mm\parskip=0mm\topsep=0pt\leftmargin=2mm
\rightmargin=2\leftmargin\leftmargin=4pt\relax}
\item\relax}
{\endlist\end{shadedwbar}\egroup}


\title{Seminar 12 for Lecture 11 and 12}
\author{ }
\date{\today}


\begin{document}
\maketitle
\section{Question 1:}

What kind of interface can be used to automate tests?

\begin{itemize}
 \item \textbf{a)} \textit{API}
 \item \textbf{b)} \textit{GUI}
 \item \textbf{c)} \textit{Both API and GUI}
 \item \textbf{d)} \textit{None of the above}
\end{itemize}


\underline{\textbf{Answer: c}}. Both. (Ref. lecture 11, p. 6.)\\


\noindent Comments:\\
\framebox(450,100){}\\

\section{Question 2:} Which of the following are advantages of test automation? \\

\begin{itemize}
 \item \textbf{a)} \textit{Tests run faster and can be more complex}
 \item \textbf{b)} \textit{Tests are run by machines and the results are interpreted by humans}
 \item \textbf{c)} \textit{Data sets used in testing can be very simple}
 \item \textbf{d)} \textit{The results of running the tests is always the same}
\end{itemize}

\underline{\textbf{Answer: a}}. (Comments. b: There are also machine-interpretable results. c: not an advantage. d: makes little sense) \\


\noindent Comments:\\
\framebox(450,100){}\\

\section {Question 3:} If you use an item (image, music, etc) from the public domain:  \\

\begin{itemize}
 \item \textbf{a)} \textit{If you use an item (image, music, etc) from the public domain}
 \item \textbf{b)} \textit{You don’t have to ask for permission to use it, but you still need to cite the source}
 \item \textbf{c)} \textit{You have to find either the author or the owner and ask to for permission to use publicly the item}
 \item \textbf{d)} \textit{You have to get the public permission from the author to use the item}
\end{itemize}

\underline{\textbf{Answer: b}} 

\noindent Comments:\\
\framebox(450,100){}\\

\section {Question 4:} Which of the following are valid strategies for choosing a certain type of license for your software?\\

\begin{itemize}
 \item \textbf{I)} \textit{Presence/absence of competition on the market}
 \item \textbf{II)} \textit{Remuneration hoped to be realized over the term of the license [Remuneration: money paid for work or a service]}
 \item \textbf{III)} \textit{Legal, political, and cultural environments}
\end{itemize}

\underline{\textbf{Answer: d: I, II and III}} \\

\noindent Comments:\\
\framebox(450,100){}\\

\section {Question 5: Open-end question} GUI testing or API testing? Which one do you prefer?
Ex: should a mechanic drive your car to test it, or do you trust him hooking your car to a testing device? \\

\noindent Comments:\\
\framebox(450,100){}\\

(Prefer API? Not all software implementations have a GUI)

\section {Question 6: Open-end question} Can you give examples of situations in which you were allowed to use items you found on the internet in your university projects – but that wouldn’t be allowed in your job in a commercial company?


\noindent Comments: \heartsuit Raluca \\
\framebox(450,100){}\\

(I used an image from flickr.com of the Creative Commons (CC-nc) non-commercial license)

\end{document}