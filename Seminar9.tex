\documentclass[12pt,a4paper,norsk]{article}

\usepackage{fancyvrb}
\usepackage[T1]{fontenc}
\usepackage[utf8]{inputenc}
\usepackage{fancyvrb}
\usepackage{framed}
\usepackage{color}
\usepackage[colorlinks]{hyperref}
\hypersetup{linkcolor=DarkRed}
\providecommand{\shadedwbar}{}
\definecolor{shadecolor}{rgb}{0.87843, 0.95686, 1.0}
\renewenvironment{shadedwbar}{
\def\FrameCommand{\color[rgb]{0.7,     0.95686, 1}\vrule width 1mm\normalcolor\colorbox{shadecolor}}\FrameRule0.6pt
\MakeFramed {\advance\hsize-2mm\FrameRestore}\vskip3mm}{\vskip0mm\endMakeFramed}
\providecommand{\shadedquoteBlueBar}{}
\renewenvironment{shadedquoteBlueBar}[1][]{
\bgroup\rmfamily
\fboxsep=0mm\relax
\begin{shadedwbar}
\list{}{\parsep=-2mm\parskip=0mm\topsep=0pt\leftmargin=2mm
\rightmargin=2\leftmargin\leftmargin=4pt\relax}
\item\relax}
{\endlist\end{shadedwbar}\egroup}


\title{Seminar 9 for Lecture 8}
\author{ }
\date{\today}


\begin{document}
\maketitle
\section{Question 1:}
Which is the role of personas in the accessibility studies?

\begin{itemize}
 \item \textbf{a)} \textit{They provide laws, regulations and guidelines for programmers to follow.}
 \item \textbf{b)} \textit{They give a realistic view of the people we design for.}
 \item \textbf{c)} \textit{They are the main stakeholders for any software being developed.}
 \item \textbf{d)} \textit{They test that the software has all quality certifications in place.}
\end{itemize}


\underline{\textbf{Answer: b}}. They give a realistic view of the people we design for.\\


\noindent Comments:\\
\framebox(450,100){}\\

\section{Question 2:} Which of the following assistive technologies is mainly used by blind users? \\

\begin{itemize}
 \item \textbf{a)} \textit{Screen reader.}
 \item \textbf{b)} \textit{Split keyboard.}
 \item \textbf{c)} \textit{Communication access real-time translation.}
 \item \textbf{d)} \textit{Text configuration tools and settings.}
\end{itemize}

\underline{\textbf{Answer: a}}. Screen reader. Note: Braille display/terminal could also be useful to blind people. \\


\noindent Comments:\\
\framebox(450,100){}\\

\section {Question 3:} Which of the following are the four principles of accessible web-content?  \\

\begin{itemize}
 \item \textbf{a)} \textit{Security, interoperability, accuracy, maintainability.}
 \item \textbf{b)} \textit{High-contrast, high-performance, responsiveness, accuracy.}
 \item \textbf{c)} \textit{Easy to remember, easy to like, easy to modify, easy to maintain.}
 \item \textbf{d)} \textit{Perceivable, operable, understandable, robust.}
\end{itemize}

\underline{\textbf{Answer: d}} Ref. recommendations in Web Content Accessibility Guidelines (WCAG) 2.0: «At the top are four principles that provide the foundation for web accessibility» \\

\noindent Comments:\\
\framebox(450,100){}\\

\section {Question 4:} What does it mean that a user interface is operable?\\

\begin{itemize}
 \item \textbf{I)} \textit{All functionality is available from a keyboard.}
 \item \textbf{II)} \textit{All user interface is navigable with a keyboard.}
 \item \textbf{III)} \textit{All functionality can be operated by the same person.}
\end{itemize}

\underline{\textbf{Answer: a: I and II}} Ref. p. 25, Lecture 8: Operable - User interface components and navigation must be operable.\\

\noindent Comments:\\
\framebox(450,100){}\\

\section {Question 5: Open-end question} Compare the keyboard accessibility for the following websites:
www.google.com; www.bing.com \\

\noindent Comments:\\
\framebox(450,100){}\\

\section {Question 6: Open-end question} Which do you think it’s the easiest to forget about?

\begin{itemize}
 \item Non-English speaker
 \item Deaf-mute
 \item Blind
 \item ARMG (age-related macular degeneration)
\end{itemize}

Wolfram-Alpha query 'English total number of speakers' 760 millions (guesstimate) \\
\url{http://en.wikipedia.org/wiki/Languages_used_on_the_Internet}: Top 10 million websites, 55 pst in English \\
Estimates of number of Internet users by language as of 31 May 2011: No 1: 27 pst English, No 2: 25 pst Chinese, No 3: 8 pst Spanish, ..., 11-36: 17 pst others


\noindent Comments:\\
\framebox(450,100){}\\

\end{document}